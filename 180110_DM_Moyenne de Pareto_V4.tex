\documentclass[a4paper,11pt]{article}
\usepackage[T1]{fontenc}
\usepackage[french]{babel}
\usepackage{amsmath} 
\usepackage{dsfont}
\usepackage{amsthm,amsfonts,amssymb,epsfig,titling,caption,graphicx,url,array}
\usepackage[left=2cm,right=2cm,top=2cm,bottom=2cm]{geometry}

\begin{document}

\noindent\Large{Janvier 2017 - Statistique 1}
\hspace{60mm}
\Large{ENSAE ParisTech}

\vspace{70mm}
\begin{center} \textsc{\Huge{Projet de Statistiques \\ Moyenne d'une Loi de Pareto}}\\
\vspace{5mm}
\Large{Sujet propos\'e par C. Champagne}
\end{center}
\vspace{80mm}
\begin{flushright}\Large{Cl\'ementine Rosier - Romain Fichou}
\end{flushright}
\newpage

On dispose de n observations $(X_{1},...,X_{n})$ iid tir\'ees selon une loi de Pareto, de fonction de r\'epartition $F(x)=1-(\frac{c}{x})^{\theta}$, pour $x>c$.

\subsubsection*{Question 1}
Soit $u\in \mathclose{[}0;1\mathopen{[}$, on a :
$$1-(\frac{c}{x})^{\theta}=u <=> x=c(1-u)^{-\frac{1}{\theta}}$$
Donc la fonction de r\'epartition inverse est donn\'ee par : $G(u)=c(1-u)^{-\frac{1}{\theta}}$, pour $u\in \mathclose{[}0;1\mathopen{[}$.

%%Ins�rer code R et histogrammes

\subsubsection*{Question 2}
Cherchons tout d'abord si la loi de Pareto admet une esp\'erance et si oui, sous quelle conditions. 
\\ Soit $c\in\mathbb{R}$, en notant$f$ la densit\'e de la loi, on a:
$$f(x)=\theta\frac{c^{\theta}}{x^{\theta+1}} \mathds{1}_{x>c}$$

Ainsi, on calcule:

\begin{align*}
\mathbb{E}(|x|)&= \int_{-\infty}^{+\infty}|x|\theta\frac{c^{\theta}}{x^{\theta+1}} \mathds{1}_{x>c} dx 
\\ &= \theta c^{\theta}  \int_{c}^{+\infty} \frac{|x|}{x^{\theta + 1}} dx
\end{align*}

 Or on sait que pour tout $t\neq0$:
 \begin{center}
 	$\int_{t}^{+\infty}\frac{1}{x^{\theta}}dx$ converge si et seulement si $ \theta>1$.
 \end{center}
Et pour tout $c$ et $t$ r\'eels, $\int_{c}^{t}\frac{1}{x^{\theta}}dx$ converge \\
Donc pour tout $c$ r\'eel, $\mathbb{E}(|x|)<+\infty$ si et seulement si $\theta<1$.\\ 
Ainsi la loi de Pareto admet une esp\'erance si et seulement $\theta>1$.
\\ Alors pour tout $c$ r\'eel et pour tout $\theta$ strictement sup\'erieur \`a 1, on peut calculer:\\

\begin{align*}
\mathbb{E}(x)&= \int_{c}^{+\infty} \theta c^{\theta} \frac{1}{x^{\theta}}dx\\
			&= \theta c^{\theta}\times [-\frac{1}{1+\theta}x^{1+\theta}]^{+\infty}_{c}\\
			&= \theta c^{\theta} \frac{-1}{1 - \theta}\times c^{1-\theta} +0\\
\textnormal{Ainsi, } \mathbb{E}(x)&= \frac{\theta}{\theta -1}c
\end{align*}

\section*{PARTIE I : Cadre fr\'equentiste}
\subsection*{A. Maximum de vraisemblance}
\subsubsection*{Question 1}
Calculons tout d'abord la vraisemblance. On a, comme les n observations sont ind\'ependantes et identiquement distrbu\'ees:
$$L_{n}(x_{1},...,x_{n};\theta)= \prod_{i=1}^{n} \frac{\theta c^{\theta}}{x_{i}^{\theta +1}} = \theta^{n}c^{\theta n}\times ( \prod_{i=1}^{n} x_{i})^{-(\theta +1)}$$
On passe \`a la log-vraisemblance pour faciliter ensuite la maximisation. En effet, la fonction logarithme \'etant croissante, les deux maximisation sont \'equivalentes. On obtient alors:
$$l_{n}(x_{1};...;x_{n};\theta)= n\ln\theta + n\theta\ln(c)-(\theta+1)\sum_{i=1}^{n}\ln(x_{i})$$

La condition de premier ordre, nous donne alors:
$$\frac{\partial l_{n}}{\partial \theta}(x_{1};...;x_{n};\theta)=0 $$
i.e. $$\frac{n}{\theta} + n\ln(c) - \sum_{i=1}^{n}\ln(x_{i})=0$$
i.e.$$ \frac{n}{\theta}= \sum{i=1}^{n}\ln(x_{i})-n\ln(c)$$
i.e.$$\theta = \frac{n}{\sum_{i=1}^{n}\ln(\frac{x_{i}}{c})}$$
\\ De plus la d\'eriv\'ee seconde de la log-vraisemblance sur $\theta$, pour tout $\theta$:
$$\frac{\partial^{2} l_{n}}{\partial^{2} \theta}(x_{1};...;x_{n};\theta)=\frac{-n}{\theta^{2}} <0 $$
Donc le point trouv\'e pr\'ec\'edemment est bien un maximum.

Ainsi l'estimateur du maximum de vraisemblance est:
$$\theta^{emv}=\frac{n}{\sum_{i=1}^{n}\ln(\frac{x_{i}}{c})}$$

\subsubsection*{Question 2}
Dans le cas, $\theta < 1$, on cherche un estimateur param\`etrique de l'esp\'erance m de la loi de Pareto. On prend $c=1$. On a:
\begin{align*}
\hat{m} &= \frac{\theta ^{emv}}{\theta ^{emv}-1}\times 1\\
		&= \frac{\frac{n}{\sum_{i=1}^{n}\ln(\frac{x_{i}}{c})}}{\frac{n}{\sum_{i=1}^{n}\ln(\frac{x_{i}}{c})} -1}\\
		&= \frac{n}{n-\sum_{i=1}^{n}\ln(\frac{x_{i}}{c})}
\end{align*}
\\ Cherchons maintenant la loi asymptotique de cet estimateur.\\
On se concentre tout d'abord sur l'estimateur $\theta ^{emv}$. On note $f$ la densit\'e de la loir de Pareto. On a pour tout $x>c$,:
$$f(x;\theta)=\theta\frac{c^{\theta}}{x^{\theta+1}}$$
Calculons l'information de Fisher de $\theta$:
\begin{align*}
I_{n}(\theta)&= - \mathbb{E}_{\theta}[\frac{\partial^{2} l_{n}}{\partial^{2} \theta}(x_{1};...;x_{n};\theta)]\\
		&= \mathbb{E}_{\theta}[\frac{n}{\theta^{2}}]\\
		&= \frac{n}{\theta^{2}}
\end{align*}


On remarque donc que:
\begin{enumerate}
	\item Le support de $f(.,\theta)$ est le m\^eme pour tout $\theta$ puisqu'il ne d\' epend que de $c$.
	\item $\theta_{0} (=3)$ est un point int\'erieur de $\Theta$
	\item Pour tout $x$, et \`a proximit\'e de $\theta_{0}$, $f$ est trois fois d\'erivale en $\theta$ et vaut $\frac{2}{\theta^{3}}$ que l'on peut majorer par la fonction constante \' egale \`a 2. De plus si on pose $M:x\longmapsto 2$, on a $\mathbb{E}_{\theta_{0}}(M(X))=2<+\infty$
	\item On a $I_{1}(\theta)= \frac{1}{\theta^{2}}>0$
\end{enumerate}
Ainsi, par th\'eor\^eme, on a que la solution de  l'\'equation de vraisemblance $\theta^{emv}$ v\'erifie:
$$\sqrt{n}(\theta^{emv}-\theta_{0})\stackrel{\mathcal{L}}{\longrightarrow}\mathcal{N}(0,\theta^{2}) \ \  [\mathbb{P}_{\theta_{0}}]$$
\\ On veut ensuite appliquer la Delta-method pour obtenir la loi asymptotique de l'estimateur de $m$.
\\Posons $g:x \longmapsto \frac{x}{1-x}$ d\'efini sur $\mathopen{]}1\,;+\infty\mathclose{]}$.
\\ On a, pour tout $\theta>1$:
$$g'(\theta)=\frac{1}{(\theta-1)^{2}}\neq0$$
Et comme, $\hat{m}=g(\theta^{emv})$, on sait que le $\hat{m}$ ainsi d�fini est un estimateur du maximum de vraisemblance de $m$, et on obtient:
$$\sqrt{n}(g(\theta^{emv})-g(\theta_{0}))\stackrel{\mathcal{L}}{\longrightarrow}\mathcal{N}(0,g'(\theta)^{2}\theta^{2}) \ \  [\mathbb{P}_{\theta_{0}}]$$
D'o\`u:
$$\sqrt{n}(\hat{m}-m)\stackrel{\mathcal{L}}{\longrightarrow}\mathcal{N}(0,\frac{\theta^{2}}{(\theta-1)^{4}}) \ \  [\mathbb{P}_{\theta_{0}}]$$


\subsubsection*{Question 3}
D'apr\`es le r\'esultat de la question pr\'ec\'edente, on a :
$$\sqrt{n}(\hat{m}-m)\stackrel{\mathcal{L}}{\longrightarrow}\mathcal{N}(0,\frac{\theta^{2}}{(\theta-1)^{4}}) \ \  [\mathbb{P}_{\theta_{0}}]$$

		
De plus, on sait que $\theta^{emv}\stackrel{p.s.}{\longrightarrow}\theta\ \  [\mathbb{P}_{\theta_{0}}] $ .
\\Donc, on a: 
$$\sqrt{\frac{(\theta^{emv}-1)^{4}}{(\theta^{emv})^{2}}} \stackrel{p.s.}{\longrightarrow}  \sqrt{\frac{(\theta-1)^{4}}{(\theta)^{2}}}\ \  [\mathbb{P}_{\theta_{0}}] $$

Or, la convergence presque s\^ure entraine la convergence en probabilit\'e.

On obtient ainsi:

$$\frac{(\theta^{emv}-1)^{2}}{\theta^{emv}} \stackrel{\mathbb{P}}{\longrightarrow}  \frac{(\theta-1)^{2}}{\theta}\ \  [\mathbb{P}_{\theta_{0}}] $$


Enfin, la fonction $h:(x,y)\longmapsto xy$ est une fonction continue sur $\mathbb{R}$.

On peut donc appliquer le th\' eor\`eme de Slutsky et obtenir le r\'esultat suivant:

$$\frac{(\theta^{emv}-1)^{2}}{\theta^{emv}}\times s\sqrt{n}(\hat{m}-m)\stackrel{\mathcal{D}}{\longrightarrow}\mathcal{N}(0,1) \ \  [\mathbb{P}_{\theta_{0}}]$$

On peut alors construire un intervalle de confiance asymptotique pour le seuil $\alpha$, en notant $u_{\alpha} $ le quantile $(1-\frac{\alpha}{2})$ d'une loi normale centr\'ee r\'eduite. En effet:
\begin{align*}
1-\alpha &= \lim_{n\rightarrow =\infty}\mathbb{P}(-u_{\alpha}\leq\frac{(\theta^{emv}-1)^{2}}{\theta^{emv}}\times\sqrt{n}(\hat{m}-m)\leq u_{\alpha})\\
		&= \lim_{n\rightarrow =\infty}\mathbb{P}(\hat{m}+\frac{u_{\alpha}\theta^{emv}}{\sqrt{n}(\theta^{emv}-1)^{2}}\geq m \geq \hat{m}-\frac{u_{\alpha}\theta^{emv}}{\sqrt{n}(\theta^{emv}-1)^{2}})
\end{align*}


On obtient ainsi l'intervalle de confiance suivant:
\begin{align*}
&[\hat{m}-\frac{u_{\alpha}\theta^{emv}}{\sqrt{n}(\theta^{emv}-1)^{2}}; \hat{m}+\frac{u_{\alpha}\theta^{emv}}{\sqrt{n}(\theta^{emv}-1)^{2}}]
\\&=[\hat{m}(1-\frac{u_{\alpha}}{\sqrt{n}(\theta^{emv}-1)}); \hat{m}(1+\frac{u_{\alpha}}{\sqrt{n}(\theta^{emv}-1)})]\\
		&=[\frac{1}{1-\frac{\sum_{i=1}^{n}\ln(\frac{x_{i}}{c})}{n}}(1-\frac{u_{\alpha}}{\sqrt{n}(\frac{n}{\sum_{i=1}^{n}\ln(\frac{x_{i}}{c})}-1)}); \frac{1}{1-\frac{\sum_{i=1}^{n}\ln(\frac{x_{i}}{c})}{n}}(1+\frac{u_{\alpha}}{\sqrt{n}(\frac{n}{\sum_{i=1}^{n}\ln(\frac{x_{i}}{c})}-1)})]		
\end{align*}

\subsubsection*{Question 4}
%Faire les calculs avec les donn�es simul�es
%Pour $\alpha = 0,05$, on a $u_{\alpha}=1,96$. On obtient ainsi:

\subsection*{B. M\'ethode des moments}
\subsubsection*{Question 1}
Par d\'efinition de m, on sait que :
\begin{center}
$\mathbb{E}(X_{1}) = m$
\end{center}
Donc un estimateur de la moyenne m par la m\'ethode des moments est naturellement la moyenne empirique :
\begin{center}
$\widehat{m_{n}}=\overline{X_{n}}=\frac{1}{n} \sum_{i=1}^{n}X_{i}$
\end{center}

\subsubsection*{Question 2}
Dans le cadre de l'estimation par la m\'ethode des moments, on a existence d'une loi asymptotique de l'estimateur si $\mathbb{V}(X_{1}) < + \infty$. Or, comme $X_{1}$ suit une loi de Pareto, cette variance existe si et seulement si $\theta > 2$ et vaut alors :
\begin{center}
$\mathbb{V}(X_{1}) = \frac{\theta}{(\theta-1)^{2}(\theta-2)}c^{2}$
\end{center}
Sous cette condition d'existence, on a alors : 
\begin{center}
$\sqrt{n}(\frac{1}{n}\sum_{i=1}^{n}X_{i} - m) \underset{n\to+\infty}{\overset{\mathcal{L}}{\longrightarrow}} \mathcal{N}(0,\mathbb{V}(X_{1})) \ \  [\mathbb{P}_{\theta}]  $
\end{center}
Ou encore en prenant l'expression de la variance avec $c=1$ :
\begin{center}
$\sqrt{n}(\frac{1}{n}\sum_{i=1}^{n}X_{i} - m) \underset{n\to+\infty}{\overset{\mathcal{L}}{\longrightarrow}} \mathcal{N}(0,\frac{\theta}{(\theta-1)^{2}(\theta-2)}) \ \  [\mathbb{P}_{\theta}] $
\end{center}

\subsubsection*{Question 3}
Gr\^ace a cette loi asymptotique, on peut d\'eduire un intervalle de confiance pour m. On va pour cela utiliser le fait que l'estimateur obtenu par la m\'ethode des moments est fortement convergent : 
$$ \widehat{m_{n}} \underset{n\to+\infty}{\overset{p.s.}{\longrightarrow}} m \ \ [\mathbb{P}_{\theta}]  $$
Pour construire l'intervalle de confiance, il nous faut un estimateur fortement convergent de $\theta$ qui est inconnu. En utilisant la question 2 (avec $c=1$), on peut exprimer $\theta$ en fonction de m :
$$ m = \frac{1\times\theta}{\theta-1} $$
i.e. $$ m(\theta - 1) = \theta $$
i.e. $$ (1 - m)\theta = -m $$
i.e. $$ \theta = \frac{m}{m-c} $$
Comme la fonction $ x \longmapsto \frac{x}{x-1}$ est continue sur $]1;+\infty[$ ($\theta > 1$ pour que l'esp\'erance existe), $\widehat{\theta_{n}} = \frac{\widehat{m_{n}}}{\widehat{m_{n}}-1}$ est alors un estimateur fortement convergent de $\theta$. \\
Donc, on a : 
$$\sqrt{\frac{\widehat{\theta_{n}}}{(\widehat{\theta_{n}}-1)^{2}(\widehat{\theta_{n}}-2)}} \underset{n\to+\infty}{\overset{p.s.}{\longrightarrow}} \sqrt{\frac{\theta}{(\theta-1)^{2}(\theta-2)}} \ \ [\mathbb{P}_{\theta}]  $$
Or, la convergence presque s\^ure entra\^ine la convergence en probabilit\'e. Donc en d�veloppant les calculs, on obtient :

$$\sqrt{\frac{\frac{\widehat{m_{n}}}{\widehat{m_{n}}-1}}{(\frac{\widehat{m_{n}}}{\widehat{m_{n}}-1}-1)^{2}(\frac{\widehat{m_{n}}}{\widehat{m_{n}}-1}-2)}} 
\underset{n\to+\infty}{\overset{\mathbb{P}}{\longrightarrow}} 
\sqrt{\frac{\theta}{(\theta-1)^{2}(\theta-2)}} \ \ [\mathbb{P}_{\theta}] $$

i.e. $$ \sqrt{\frac{\widehat{m_{n}}(\widehat{m_{n}}-1)^{2}}{2-\widehat{m_{n}}}}
\underset{n\to+\infty}{\overset{\mathbb{P}}{\longrightarrow}} 
\sqrt{\frac{\theta}{(\theta-1)^{2}(\theta-2)}} \ \ [\mathbb{P}_{\theta}] $$

Enfin, la fonction $ (x,y)\longmapsto xy$ est une fonction continue sur $\mathbb{R}$. On peut donc appliquer le th\' eor\`eme de Slutsky et obtenir le r\'esultat suivant:

$$\sqrt{\frac{2-\widehat{m_{n}}}{\widehat{m_{n}}(\widehat{m_{n}}-1)^{2}}} \times \sqrt{n}(\widehat{m_{n}} - m)
\underset{n\to+\infty}{\overset{\mathcal{L}}{\longrightarrow}} \mathcal{N}(0,1) \ \  [\mathbb{P}_{\theta}] $$

On peut alors construire un intervalle de confiance asymptotique pour le seuil $\alpha$, en notant $u_{\alpha} $ le quantile $(1-\frac{\alpha}{2})$ d'une loi normale centr\'ee r\'eduite. En effet:
\begin{align*}
1-\alpha &= \lim_{n\rightarrow +\infty}\mathbb{P}(-u_{\alpha} \leq \sqrt{\frac{2-\widehat{m_{n}}}{\widehat{m_{n}}(\widehat{m_{n}}-1)^{2}}} \times \sqrt{n}(\widehat{m_{n}} - m) \leq u_{\alpha}) \\
	&= \lim_{n\rightarrow +\infty}\mathbb{P}(\widehat{m_{n}}+u_{\alpha}\times\sqrt{\frac{\widehat{m_{n}}(\widehat{m_{n}}-1)^{2}}{n(2-\widehat{m_{n}})}} \geq  m \geq \widehat{m_{n}}-u_{\alpha}\times\sqrt{\frac{\widehat{m_{n}}(\widehat{m_{n}}-1)^{2}}{n(2-\widehat{m_{n}})}}) 
\end{align*}

On obtient ainsi l'intervalle de confiance suivant:
\begin{align*}
&[\widehat{m_{n}}-u_{\alpha}\times\sqrt{\frac{\widehat{m_{n}}(\widehat{m_{n}}-1)^{2}}{n(2-\widehat{m_{n}})}};\widehat{m_{n}}+u_{\alpha}\times\sqrt{\frac{\widehat{m_{n}}(\widehat{m_{n}}-1)^{2}}{n(2-\widehat{m_{n}})}}]
\\ &= [\frac{1}{n} \sum_{i=1}^{n}X_{i}  - u_{\alpha}\times \sqrt{\frac{\frac{1}{n} \sum_{i=1}^{n}X_{i}(\frac{1}{n} \sum_{i=1}^{n}X_{i}-1)^{2}}{2n-\sum_{i=1}^{n}X_{i}}} ; \\
&\frac{1}{n} \sum_{i=1}^{n}X_{i}  + u_{\alpha}\times \sqrt{\frac{\frac{1}{n} \sum_{i=1}^{n}X_{i}(\frac{1}{n} \sum_{i=1}^{n}X_{i}-1)^{2}}{2n-\sum_{i=1}^{n}X_{i}}}]
\end{align*}

\subsubsection*{Question 4}
%Faire les calculs avec les donn�es simul�es
%Pour $\alpha = 0,05$, on a $u_{\alpha}=1,96$. On obtient ainsi:

\section*{PARTIE II : Cadre bay\'esien}
On se place dans le cadre bay\'esien. On suppose une distribution a priori gaussienne pour $m$, d'esp\'erance $\mu$ et de variance $\sigma^{2}$.

\subsubsection*{Question 1}

La loi a priori de $m$ s'\'ecrit sous la forme suivante:
$$\pi (m) = \frac{1}{\sqrt{2\pi\sigma^{2}}}\exp(-\frac{(m-\mu)^{2}}{2\sigma^{2}})$$
De plus, la vraisemblance en fonction de $\theta$ s'\'ecrit:
$$L_{n}(x_{1},...,x_{n};\theta) = \theta^{n}1^{\theta n}\times ( \prod_{i=1}^{n} x_{i})^{-(\theta +1)}$$

Or $\theta=\frac{m}{m-1}$ (cf:question 3).\\
D'o\`u:
$$L_{n}(x_{1},...,x_{n};m)= (\frac{m}{m-1})^{n}\times ( \prod_{i=1}^{n} x_{i})^{-(\frac{m}{m-1} +1)}$$

On trouve ainsi:
$$\pi(m|x)\propto (\frac{m}{m-1})^{n}\times ( \prod_{i=1}^{n} x_{i})^{-(\frac{m}{m-1})}\exp(-\frac{(m-\mu)^{2}}{2\sigma^{2}})$$

%justifier el passage au log cf ply de coraline
On passe au logarithme:
$$\ln (\pi (m|x))= n\ln(\frac{m}{m-1})+ \frac{m}{m-1}\times \sum_{i=1}^{n}\ln(x_{i})- \frac{(m-\mu)^2}{2\sigma^2}+ constante$$

\subsubsection*{Question 2}
%R

\subsubsection*{Question 3}

\subsubsection*{Question 4}

\subsubsection*{Question 5}

On sait que l'estimateur de Bayes associ\'e \`a la perte quadratique est l'\' esperance a posteriori.
On prend donc la moyenne de l'�chantillon simul\' e dans la question pr\'ec\'edente.

\end{document}
